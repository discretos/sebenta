\chapter*{Programa da disciplina}

\section*{Descrição}

Esta displicina é uma introdução à conceitos de matemática discreta e estruturas discretas tal como são utilizadas em Ciência
da Computação. As técnicas apresentadas no curso permitem aos estudantes aplicar o pensamento lógico e matemático
na resolução de problemas. Os tópicos incluem: lógica proposicional e de predicados, funções, relações, conjuntos 
e técnicas de demonstração. De acordo a disponibilidade de tempo, os seguintes tópicos também serão apresentados: grafos, árvores, computabilidade, etc.

\section*{Objectivos}


Ao completar a disciplina, os estudantes serão capazes de:
\begin{itemize}
  \item Aplicar métodos formais de lógica de proposicional e de predicados
  \item Descrever a importância e limitações da lógica de predicados
  \item Utilizar demonstrações lógicas para resolver problemas
  \item Desenvolver algoritmos recursivos baseados em indução matemática
  \item Explicar a terminologia básica das funções, relações e conjuntos
  \item Descrever como métodos formais de lógica simbólica são utilizados para modelar algoritmos reais
  \item Perceber os conceitos básicos sobre a teoria dos grafos e algoritmos relacionados\
\end{itemize} 

No geral, espera-se que os estudantes sejão capazes de aplicar estes métodos em outros tópicos do curso de Ciência da Computação
tais como desenho e análise de algoritmos e engenharia de \emph{software}.


\section*{Tópicos}
\begin{enumerate}
  \item Lógica formal
  \item Demonstrações, recorrência e análise de algoritmos
  \item Conjuntos e combinatória
  \item Relações, funções e matrizes
  \item Gráfos e árvores
  \item Álgebra de Boole e lógica computacional
\end{enumerate}

\section*{Avaliação}%
\begin{itemize}
  \item Exercícios: 10\%
  \item Primeira prova: 20\%
  \item Segunda prova: 30\%
  \item Exame final: 40\%
\end{itemize}

Os exercícios serão fornecidos nas aulas ou publicados no \emph{website} da cadeira. Os estudantes são fortemente encorajados
a resolve-los pois os mesmos irão ajudar a entender melhor os tópicos tratados nas aulas.
%
\section*{Pré-requisitos}

Conhecimentos básicos de lógica e de simbolização matemática.

\section*{Regras}

\begin{itemize}
  \item Requere-se que os estudantes leiam os acetatos/fascículos antes das aulas
  \item {A participação nas aulas é essencial para a compreensão da matéria. A assistência às aulas é de sua inteira
  responsabilidade}
  \item {Todas as provas e exames são obrigatórios, excepto por razões devidamente justificadas. Uma ausência não justificada, 
  equivale a nota zero na referida avaliação.}
\end{itemize}

\section*{Agenda (sujeita à alterações)}

\begin{table}[H]
	\centering
	\begin{tabular}{ll}%
	\toprule
	\textbf{Semana} & \textbf{Tópicos} \\ 
	\midrule
	1	&	Lógica formal\\ 
    2 	&	Demonstrações\\
    3	&	Conjuntos\\
    4	&	Funções\\
    5	&	Relações\\
    6	&	Algoritmos\\
    7	&	Indução\\
    8	&	Contagem\\
    9	&	Combinatória\\
    10	&	Recursão\\
    11	&	Grafos\\
    12	&	Algoritmos para grafos\\
    13	&	Árvores\\
    14	&	Álgebra de Boole\\
 	 \bottomrule
 	 \end{tabular}
 	 \centering
\end{table}
%Acrescentar unidades


\subsection*{Feriados e interrupções}

\begin{itemize}
  \item 17 de Setembro: Dia do Herói Nacional
\end{itemize}

\section*{Bibliografia}

\begin{table}[H]
	\begin{tabular}{ll}%
		Título & Fundamentos matemáticos para a Ciência da Computação, 6a. Edição\\
		Autor & Judith L. Gersting\\
	\end{tabular}
\end{table}


\section*{Docente}

\begin{table}[H]
	\begin{tabular}{ll}%
		Nome 			& Dizando Norton \\ 
	    Sala			& CS119, Campus Universitário	\\ 
	    Atendimento 	& Por agendamento \\
	    Telefone		& 919075391\\
	    E-mail			& \url{dizando.norton@gmail.com}\\
	    Website			& \url{http://dizan.do}\\
	    Aulas 			& Consultem o horário para a vossa turma\\
	    %				& Quarta-Feira, 7h30 - 9h50\\
	\end{tabular}
\end{table}

\section*{Página da cadeira}


O \emph{link} para página da disciplina é \url{http://www.dizan.do/t/ed2014}. A mesma contém anúncios, os acetatos e a sebenta utilizados no 
decorrer das aulas. Os estudantes são recomendados a consultar a página regularmente.

%\subsection*{Moodle}
%\subsection*{dizan.do}
%\subsection*{Dropbox}




