\chapter*{Exercícios - Conjuntos e funções}
%%1

\section*{Conjuntos}

\begin{enumerate}
  	\item Seja $A$ o conjunto dos estudantes que vivem à $5 km$ do campus e $B$ o conjunto dos estudantes que vêm 
  	de bicicleta às aulas. Descreva os estudantes em cada um dos seguintes conjuntos.
  	\begin{enumerate}
  	  	\item $A \cap B$ \item $A \cup B$ \item $A - B$ \item $B - A$
  	\end{enumerate}
  	
  	\item Liste os elementos dos seguintes conjuntos:
  	\begin{enumerate}
  		 \item $\{x | x \in \mathbb{N} \land x^2 < 25\}$ \item $\{x | x$ é um dos antigos vencedores da Fórmula 1 $\}$
  		 \item $\{x | x \in \mathbb{R} \land x^2 = -1\}$ \item $\{x | x \in \mathbb{N} \land x^2 - 5x + 6 = 0\}$
	\end{enumerate}
	
	\item Qual é cardinalidade de cada um dos seguintes conjuntos?
	\begin{enumerate}
	  	\item $ S = \{a, \{a, \{a\}\}\}$ \item $\{\{a\}, \{\{a\}\}\}$ \item $\{a, \{\emptyset \}, \emptyset\}$
	  	\item $\{\emptyset, \{\emptyset, \{\emptyset\}\},\{\emptyset, \{\emptyset, \{\emptyset\}\}\}\}$
	\end{enumerate}
	
  	\item Suponha que $A$ é o conjunto dos estudantes do segundo ano e $B$ é o conjunto dos estudantes de Lógica de Programação.
  	Descreva cada um dos conjuntos em termos de $A$ e $B$.
  	\begin{enumerate}
  		\item O conjunto dos estudantes do segundo ano com a cadeira de Lógica de Programação
  		\item O conjunto dos estudantes do segundo ano que não têm a cadeira de Lógica de Programação
  		\item O conjunto dos estudantes que são, ou do segundo ano ou têm a cadeira de Lógica de Programação
  		\item O conjunto dos estudantes que não são do segundo ano, nem têm a cadeira de Lógica de Programação
	\end{enumerate}

	\item Sejam, $A = \{a, b, c, d, e\}$ e $B = \{a, b, c, d, e, f, g, h\}$ encontre:
	\begin{enumerate}
	  \item $A \cup B$ \item $A \cap B$ \item $A - B$ \item $B - A$ 
	\end{enumerate}
	
	\item Desenhe os diagramas de Venn para cada uma das seguintes combinações dos domínios A, B e C
	\begin{enumerate}
		\item $A \cap (B \cup C)$ \item $A \cap B \cap C$ \item $(A - B) \cup (A - C) \cup (B - C)$ 
		\item $A \cap (B - C)$ \item $(A \cap B) \cup (A \cap C)$ \item $(A \cap B) \cup (A \cap C)$
	\end{enumerate}
\end{enumerate}

\section*{Funções}

\begin{enumerate}
	\item Determine se $f$ é uma função de $\mathbb{Z}$ em $\mathbb{R}$ se
	\begin{enumerate}
		\item $f(n) = \pm n$ \item $\sqrt{n^2 + 1}$ \item $\frac{1}{(n^2 - 4)}$  
	\end{enumerate}
	
	\item Determine se as seguintes funções de $\mathbb{Z}$ para $\mathbb{Z}$ são injectivas ou sobrejectivas.
	\begin{enumerate}
	  	\item $f(n) = n - 1$ \item $f(n) = n^2 + 1$ \item $f(n) = n^3$ \item $f(n) = \frac{n}{2}$
	\end{enumerate}
	
	\item Considere as seguintes funções do conjunto dos estudantes de Estruturas Discretas. Em que condições uma função é injectiva
	se ela atribui a um estudante o seu:
	\begin{enumerate}
	  	\item Número de telemóvel
	   	\item Número de estudante
	   	\item Nota final
	   	\item Local de Nascimento
	\end{enumerate}

	\begin{description}
		\item[Definição 1] Uma função diz-se \emph{bijectiva} quando é ao mesmo tempo injectiva (um para um) e sobrejectiva.
		\item[Exemplo 1] Seja $f$ uma função da forma $\{a,b,c,d\}$ para $\{1,2,3,4\}$ com $f(a) = 4, f(b) = 2, f(c) = 1$ e $f(d) = 3$.
		A função $f$ é uma função injectiva e sobrejectiva. É injectiva porque não existem valores no domínio que são mapeados
		para o mesmo valor no co-domínio. É sobrejectiva porque todos os quatro elementos do co-domínio são imagens dos elementos
		no domínio. Então $f$ é uma função \emph{bijectiva} ou uma \emph{bijeção}.	
	\end{description}
	
	\item Determina se cada uma das seguintes funções é uma bijeção de $\mathbb{R}$ para $\mathbb{R}$.
	\begin{enumerate}
		\item $f(x) = -3x + 4$ \item $f(x) = -3x^2 + 7$ \item $f(x) = (x+1)/(x+2)$ \item $f(x) = x^5 + 1$ \item $f(x) = 2x + 1$
		\item $f(x) = x^2 + 1$
	\end{enumerate}
	
	\begin{description}
		\item[Definição 2] Seja $f$ a função de $A$ para $B$ e seja $S$ um subconjunto de $A$. A imagem de $S$ sobre a função $f$ é o
		subconjunto de $B$ que consiste nas imagens dos elementos de $S$. Denotamos a imagem de $S$ por $f(S)$, tal que
		$f(S) = \{t | \exists s \in S(t = f(s))\}$. Também podemos utilizar a representação $\{f(s) | s \in S\}$
		\item[Exemplo 2] Seja $A = \{a,b,c,d,e\}$ e $B = \{1,2,3,4\}$ com $f(a)=2, f(b)=1, f(c)=4, f(d)=1$, e $f(e)=1$. A imagem
		do subconjunto $S = \{b,c,d\}$ é o conjunto $f(S) = \{1, 4\}$.
	\end{description}

	\item Seja $S = \{-1,0,2,4,7\}$ encontre $f(S)$ se
	\begin{enumerate}
	  	\item $f(x)=1$ \item $f(x)=2x + 1$ \item $f(x)=\lceil \frac{x}{5} \rceil$ \item $f(x) = \lfloor \frac{(x^2 + 1)}{3} \rfloor$  
	\end{enumerate}

	\item Seja $f(x) = \lfloor \frac{x^2}{3} \rfloor$, encontre $f(S)$ se,
	\begin{enumerate}
		\item $S = \{-2,-1,0,1,2,3\}$ \item $S=\{0,1,2,3,4,5\}$ \item $S=\{1,5,7,11\}$ \item $S=\{2,6,10,14\}$ 
	\end{enumerate}
	
	\item Seja $f: \mathbb{N} \to \mathbb{N}$ definida por $f(x) = x + 1$. Seja $g: \mathbb{N} \to \mathbb{N}$ definida por $g(x) = 3x$.
	Calcule o seguinte:
	\begin{enumerate}
	  	\item $(g \circ f)(5)$ \item $(f \circ g)(5)$ \item $(g \circ f)(x)$ \item $(f \circ g)(x)$ \item $(f \circ f)(x)$
	  	\item $(g \circ g)(x)$
	\end{enumerate}
	
	\item Para cada uma das seguintes bijeções $f: \mathbb{R} \to \mathbb{R}$, encontre $f^{-1}$
	\begin{enumerate}
	  	\item $f(x)=7x$ \item $f(x) = x^3$ \item $f(x) = \frac{(x + 4)}{3}$
	\end{enumerate}
\end{enumerate}

\vspace*{2em}
\begin{center}\textbf{Continua!!! (Mais exercícios)}\end{center}