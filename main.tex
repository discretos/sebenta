\documentclass[a4paper]{book}

\usepackage[portuguese]{babel}
\usepackage[T1]{fontenc}
\usepackage{inputenc}
\usepackage{amsmath}
%\usepackage{graphicx}
\usepackage{booktabs}
\usepackage[colorinlistoftodos]{todonotes}
\usepackage[plainpages=false,unicode]{hyperref}
\usepackage{tikz}
\usetikzlibrary{arrows}


%FROM SEPA
%\usepackage{latexsym}
\usepackage{amsmath, amsthm, amssymb, amsfonts, amsbsy}
%\usepackage{float}
\usepackage{epsfig,float,graphicx}
%\usepackage{pdfpages} 
%\usepackage{pifont}
%\restylefloat{table}
%%

%\title{Matemática Discreta - Programa da disciplina}
%\author{Dizando Norton, MSc}
%\date{\today}

%----------------------------------------------------------------------------------------
%	TITLE PAGE
%----------------------------------------------------------------------------------------

\newcommand*{\titleGM}{\begingroup % Create the command for including the title page in the document
\hbox{ % Horizontal box
\hspace*{0.2\textwidth} % Whitespace to the left of the title page
\rule{1pt}{\textheight} % Vertical line
\hspace*{0.05\textwidth} % Whitespace between the vertical line and title page text
\parbox[b]{0.75\textwidth}{ % Paragraph box which restricts text to less than the width of the page

{\noindent\Huge\bfseries Estrutura Discreta}\\[4\baselineskip] % Title
%{\large {Programa da disciplina}}\\[4\baselineskip] % Tagline or further description
{\Large \textsc{Dizando Norton}\\[1\baselineskip} % Author name
{\large (dizando.norton@gmail.com)}

\vspace{0.5\textheight} % Whitespace between the title block and the publisher
{\small \noindent {DEI - Ciência da Computação}}\\[\baselineskip] % Publisher and logo
{\small \textsc{Faculdade de Ciências - Universidade Agostinho Neto}}
}}
\endgroup}

\begin{document}
%\maketitle
\pagestyle{empty} % Removes page numbers

\titleGM % This command includes the title page

\chapter*{Programa da disciplina}

\section*{Descrição}

Esta displicina é uma introdução à conceitos de matemática discreta e estruturas discretas tal como são utilizadas em Ciência
da Computação. As técnicas apresentadas no curso permitem aos estudantes aplicar o pensamento lógico e matemático
na resolução de problemas. Os tópicos incluem: lógica proposicional e de predicados, funções, relações, conjuntos 
e técnicas de demonstração. De acordo a disponibilidade de tempo, os seguintes tópicos também serão apresentados: grafos, árvores, computabilidade, etc.

\section*{Objectivos}


Ao completar a disciplina, os estudantes serão capazes de:
\begin{itemize}
  \item Aplicar métodos formais de lógica de proposicional e de predicados
  \item Descrever a importância e limitações da lógica de predicados
  \item Utilizar demonstrações lógicas para resolver problemas
  \item Desenvolver algoritmos recursivos baseados em indução matemática
  \item Explicar a terminologia básica das funções, relações e conjuntos
  \item Descrever como métodos formais de lógica simbólica são utilizados para modelar algoritmos reais
  \item Perceber os conceitos básicos sobre a teoria dos grafos e algoritmos relacionados\
\end{itemize} 

No geral, espera-se que os estudantes sejão capazes de aplicar estes métodos em outros tópicos do curso de Ciência da Computação
tais como desenho e análise de algoritmos e engenharia de \emph{software}.


\section*{Tópicos}
\begin{enumerate}
  \item Lógica formal
  \item Demonstrações, recorrência e análise de algoritmos
  \item Conjuntos e combinatória
  \item Relações, funções e matrizes
  \item Gráfos e árvores
  \item Álgebra de Boole e lógica computacional
\end{enumerate}

\section*{Avaliação}%
\begin{itemize}
  \item Exercícios: 10\%
  \item Primeira prova: 20\%
  \item Segunda prova: 30\%
  \item Exame final: 40\%
\end{itemize}

Os exercícios serão fornecidos nas aulas ou publicados no \emph{website} da cadeira. Os estudantes são fortemente encorajados
a resolve-los pois os mesmos irão ajudar a entender melhor os tópicos tratados nas aulas.
%
\section*{Pré-requisitos}

Conhecimentos básicos de lógica e de simbolização matemática.

\section*{Regras}

\begin{itemize}
  \item Requere-se que os estudantes leiam os acetatos/fascículos antes das aulas
  \item {A participação nas aulas é essencial para a compreensão da matéria. A assistência às aulas é de sua inteira
  responsabilidade}
  \item {Todas as provas e exames são obrigatórios, excepto por razões devidamente justificadas. Uma ausência não justificada, 
  equivale a nota zero na referida avaliação.}
\end{itemize}

\section*{Agenda (sujeita à alterações)}

\begin{table}[H]
	\centering
	\begin{tabular}{ll}%
	\toprule
	\textbf{Semana} & \textbf{Tópicos} \\ 
	\midrule
	1	&	Lógica formal\\ 
    2 	&	Demonstrações\\
    3	&	Conjuntos\\
    4	&	Funções\\
    5	&	Relações\\
    6	&	Algoritmos\\
    7	&	Indução\\
    8	&	Contagem\\
    9	&	Combinatória\\
    10	&	Recursão\\
    11	&	Grafos\\
    12	&	Algoritmos para grafos\\
    13	&	Árvores\\
    14	&	Álgebra de Boole\\
 	 \bottomrule
 	 \end{tabular}
 	 \centering
\end{table}
%Acrescentar unidades


\subsection*{Feriados e interrupções}

\begin{itemize}
  \item 17 de Setembro: Dia do Herói Nacional
\end{itemize}

\section*{Bibliografia}

\begin{table}[H]
	\begin{tabular}{ll}%
		Título & Fundamentos matemáticos para a Ciência da Computação, 6a. Edição\\
		Autor & Judith L. Gersting\\
	\end{tabular}
\end{table}


\section*{Docente}

\begin{table}[H]
	\begin{tabular}{ll}%
		Nome 			& Dizando Norton \\ 
	    Sala			& CS119, Campus Universitário	\\ 
	    Atendimento 	& Por agendamento \\
	    Telefone		& 919075391\\
	    E-mail			& \url{dizando.norton@gmail.com}\\
	    Website			& \url{http://dizan.do}\\
	    Aulas 			& Consultem o horário para a vossa turma\\
	    %				& Quarta-Feira, 7h30 - 9h50\\
	\end{tabular}
\end{table}

\section*{Página da cadeira}


O \emph{link} para página da disciplina é \url{http://www.dizan.do/t/ed2014}. A mesma contém anúncios, os acetatos e a sebenta utilizados no 
decorrer das aulas. Os estudantes são recomendados a consultar a página regularmente.

%\subsection*{Moodle}
%\subsection*{dizan.do}
%\subsection*{Dropbox}





\chapter{Lógica formal}
\label{cap:logicaformal}

\section{Lógica proposicional}
\label{sec:logicaproposicional}

O que é uma prova?
Começaremos por tratar de um elemento fundamental para o entendimento da lógica - as proposições. Uma proposição é
uam sequência declarativa


\section{Lógica de predicados}
\label{sec:logicadepredicados}


\section{Exercícios}
\label{sec:exercicios1}

\begin{enumerate}
  \item Quais das seguintes frases são proposições?
  \begin{enumerate}
  	\item A lua é feita de queijo verde.
  	\item Ele é certamente, um homem alto.
  	\item Dois é um número primo.
  	\item O jovo vai acabar logo?
  	\item Os juros vão subir ano que vem.
  	\item Os juro vão descer ano que vem.
  	\item $x^2 - 4 = 0.$
  \end{enumerate}
\end{enumerate}



%Comments can be added to the margins of the document using the \todo{Here's a comment in the margin!} todo command, as shown in the example on the right. You can also add inline comments:

%\todo[inline, color=green!40]{This is an inline comment.}

\chapter*{Exercícios - Lógica formal}
%%1

\section*{Proposições}

\begin{enumerate}
  	\item Qual das seguintes sentenças é uma proposição? Quais são os valores de verdade para as que são proposições?
  	\begin{enumerate}
	  	  \item Lisboa é a capital do Brasil.
	  	  \item $2 + 3 = 5$
	  	  \item Que horas são?
	  	  \item $2^n \geq 100$
	  	  \item A lua é feita de queijo.
    \end{enumerate} 
    \item Qual é a negação de cada uma das seguintes proposições?
    \begin{enumerate}
	     \item A Joana tem um leitor de MP3.
	     \item Não existe poluição em Luanda.
	     \item $2 + 2 = 4$
	     \item O Paulo e o Tomás são amigos.
	     \item A Maria envia mais de 100 SMS por dia.
  	\end{enumerate}
  	\item Suponha que o telemóvel A tenha 256 MB de RAM e 32 GB de ROM e que a sua resolução gráfica seja de 8 MP; o telemóvel
  	B possui 288 MB de RAM, 64 GB de ROM e 4 MP de resolução gráfica; e por fim o telemóvel C possui 128 MB de RAM, 32 GB de ROM
  	e 5 MP de resolução gráfica. Determine o valor de verdade para cada uma das seguintes proposições.
  	\begin{enumerate}
  		 \item O telemóvel B possui a maior capacidade de RAM.
  		 \item O telemóvel C possui a maior capacidade de ROM ou maior resolução gráfica do que o telemóvel B.
  		 \item O telemóvel B possui mais RAM, mais ROM e mais MP do que o telemóvel A.
  		 \item Se o telemóvel B possui mais RAM e mais ROM que o telemóvel C, então também possui a maior resolução gráfica.
  		 \item O telemóvel A possui mais RAM do que o telemóvel B se e somente se o telemóvel B possui mais RAM do que o telemóvel A.
	\end{enumerate}
	\item Sejam $p$ e $q$ as proposições\\
	\hspace*{1em}$p$: Eu comprei um bilhete para o teatro esta semana\\
	\hspace*{1em}$q$: Eu ganhei um milhão de kwanzas na loteria\\
	Expresse cada uma das seguintes proposições como sentenças em Português.
	\begin{enumerate}
		  \item $\lnot p$
		  \item $p \lor q$
		  \item $p \to q$
		  \item $p \land q$
		  \item $p \leftrightarrow q$
		  \item $\lnot p \to \lnot q$
		  \item $\lnot p \land \lnot q$
		  \item $\lnot p \lor (p \lor q)$
  	\end{enumerate}
  	\item Determine se os seguintes bicondicionais são verdadeiros ou falsos.
  	\begin{enumerate}
  		\item $2+2=4$ se e somente se $1+1=2$.
  		\item $1+1=2$ se e somente se $2+3=4$.
  		\item $1+1=3$ se e somente se macacos conseguem voar.
  		\item $0 > 1$ se e somente se $2>1$.
	\end{enumerate}
  	\item Escreva cada uma das sentenças abaixo na forma ``Se $p$ então $q$''.
  	\begin{enumerate}
  	  \item É necessário lavar o carro do chefe para ser promovido.
  	  \item Quando o vento vem do sul significa que a primaveira aproxima-se.
  	  \item Uma condição suficiente para que a garantia seja válida é a de que o computador foi comprado a menos de um ano.
  	  \item O Pedro é apanhado toda vez que cabula.
  	  \item Obterás acesso ao \emph{website} se pagares a taxa de subscrição.
  	  \item Para ser eleito deves conhecer as pessoas certas.
  	\end{enumerate}
  	\item Construa a tabela de verdade para as seguintes proposições
  	\begin{enumerate}
  	  \item $p \oplus p$ \item $p \oplus \lnot p$ \item $p \oplus \lnot q$ \item $\lnot p \oplus \lnot q$
  	  \item $(p \oplus q) \lor (p \oplus \lnot q)$ \item $(p \oplus q) \land (p \oplus \lnot q)$
  	  \item $((p \to q) \to r) \to s$.
	\end{enumerate}
	\item Explique, sem utilizar uma tabela de verdade, porquê $(p \lor \lnot q) \land (q \lor \lnot r) \land (r \lor \lnot p)$ é 
	verdade quando $p, q$ e $r$ possuem os mesmos valores de verdade e falso em caso contrário.
	\item O soba de uma vila diz que existe um barbeiro numa outra vila muito distante, que apenas faz a barba à pessoas
	e somente à pessoas que não fazem a barba por sí próprias. Será que este barbeiro existe?
\end{enumerate}

\section*{Lógica proposicional}

\begin{enumerate}
	 \item Utiliza a lei de DeMorgan para encontrar a negação para cada uma das seguintes sentenças.
	 \begin{enumerate}
	   	\item O Ndongala vai procurar um emprego ou terminar a licenciatura.
	   	\item A Luísa percebe Java e Estrutura Discreta.
	   	\item O Nadilson é jovem e forte.
	   	\item A Rebecca vai viajar para o Brasil ou para a Espanha.
	 \end{enumerate}
	 \item Utilize tabelas de verdade para verificar as seguintes leis da absorção
	 \begin{enumerate}
	 	\item  $p \lor (p \land q) \equiv p$ \item $p \land (p \lor q) \equiv p$
	\end{enumerate}
	\item Mostre que $(p \to q) \land (q \to r) \to (p \to r)$ é uma tautologia
	\item Encontre uma proposição composta involvendo as variáveis $p, q$ e $r$ que seja verdadeira quando $p$ e $q$ são 
	verdadeiros e $r$ é falso, mas que seja falso em caso contrário.
	\item Encontre uma proposição composta logicamente equivalente a $p \to q$ utilizando apenas o operador $\downarrow$
\end{enumerate}


\section*{Predicados e quantificadores}

\begin{enumerate}
  	\item Seja $P(x)$ a expressão ``a palavra $x$ contém a letra $a$''. Quais são os valores de verdade para?
  	\begin{enumerate}
    	\item $P(laranja)$ \item $P(limão)$ \item $P(verdade)$ \item $P(dificil)$ \item $P(falso)$
    \end{enumerate}
    \item Seja $P(x)$ a expressão ``$x$ perde mais do que cinco horas por dia no Facebook'', onde o domínio para $x$ consiste
    em todos os estudantes. Descreve cada uma das expressões a seguir em Português.
    \begin{enumerate}
    	\item $\exists xP(x)$ \item $\forall xP(x)$ \item $\exists x \lnot P(x)$ \item $\forall x \lnot P(x)$ 
	\end{enumerate}
	\item Traduza as seguintes expressões para Português, onde $C(x)$ é ``$x$ é um comediante'' e $E(x)$ é ``$x$ é muito 
	engraçado'' e o domínio que consiste em todas as pessoas.
	\begin{enumerate}
		\item $\forall x(C(x) \to E(x))$ \item $\forall x(C(x) \land E(x))$ 
		\item $\exists x(C(x) \to E(x))$ \item $\exists x(C(x) \land E(x))$
	\end{enumerate}
	\item Seja $Q(x)$ a expressão ``$x+1 > 2$''. Se o domínio consiste em todos os números inteiros, quais são os valores
	de verdade para
	\begin{enumerate}
		\item $Q(0)$ \item $Q(-1)$ \item $Q(1)$ 
		\item $\exists xQ(x)$ \item $\forall xQ(x)$ 
		\item $\exists x \lnot Q(x)$ \item $\forall x \lnot Q(x)$
	\end{enumerate}
	
	\item Determine os valores de verdade para cada uma das seguintes afirmações se o domínio de todas as variáveis consiste
	no conjunto dos números inteiros.
	\begin{enumerate}
		\item $\forall n(n^2 \geq 0)$ \item $\exists n (n^2 = n)$ \item $\forall n(n^2 \geq n)$ \item $\exists n(n^2 < 0)$
	\end{enumerate}
	
	\item Suponha que o domínio das funções proposicionais $P(x)$ consiste nos inteiros $0, 1, 2, 3$ e $4$. Reescreva cada uma
	das proposições utilizando disjunções, conjunnções e negações.
	\begin{enumerate}
		\item $\exists xP(x)$ \item $\forall xP(x)$ \item $\exists x \lnot P(x)$ \item $\forall x \lnot P(x)$
		\item $\lnot \exists xP(x)$ \item $\lnot \forall x P(x)$
	\end{enumerate}
	
	\item Para as seguintes afirmações, encontre um domínio de tal forma que a afirmação seja verdadeira e um domínio de tal forma
	que a afirmação seja falsa.
	\begin{enumerate}
		\item Todo o mundo está a estudar Estruturas Discretas.
		\item Todos têm mais de 21 anos.
		\item Duas pessoas diferentes não possuem a mesma avó.
		\item Todo mundo fala Japonês.
		\item Alguém conhece mais do que duas pessoas.
	\end{enumerate}
	
	\item Traduza as seguintes afirmações em expressões lógicas utilizando predicados, quantificadores e conectores lógicos.
	\begin{enumerate}
	  \item Ninguém é perfeito.
	  \item Nem todo o mundo é perfeito.
	  \item Todos os teus amigos são perfeitos.
	  \item Pelo menos, um dos teus amigos é perfeito.
	  \item Alguém na tua sala foi nascido no século 21.
	  \item Tudo está no sítio certo e em perfeitas condições.
	\end{enumerate}
	
	\item Determine se $\forall x(P(x) \to Q(x))$ e $\forall xP(x) \to \forall xQ(x)$ são lógicamente equivalentes. Justifique
	a sua resposta.
\end{enumerate}

\vspace*{2em}
\begin{center}\textbf{Continua!!! (Lógica de predicados e Demonstrações)}\end{center}
\input{aulas/2.tex}
\input{aulas/2exercicios.tex}
\chapter{Conjuntos e combinatória}

\section{Conjuntos}

\begin{description}
	\item[Definição 1] (Notação de conjuntos)
\end{description}

\subsection{Propriedades dos conjuntos}

\begin{description}
	\item[Teorema 1] (Regras algebraicas dos conjuntos)
\end{description}

\begin{description}
	\item[Exemplo 1] (Diagramas de Venn e demostrações de equações de conjuntos)
\end{description}

\begin{description}
	\item[Exemplo 2] (Métodos elementares de demonstração de equações de conjuntos)
\end{description}

\begin{description}
	\item[Exemplo 3] (O método tabular para demostrações de equações de conjuntos)
\end{description}

\begin{description}
	\item[Exemplo 4] (Demonstração algebraica)
\end{description}

\subsection{Ordenação de conjuntos}

\begin{description}
	\item[Exemplo 5] (Ordem lexicográfica)
\end{description}

\begin{description}
	\item[Exemplo 6] (Ordenação por dicionário em palavras ou cadeia de caracteres)
\end{description}

\subsection{Subconjuntos}

\begin{description}
	\item[Teorema 2] (Fórmula do binómio coeficiente)
\end{description}

\begin{description}
	\item[Teorema 3] (Número de listas ordenadas)
\end{description}

\begin{description}
	\item[Exemplo 7] (Recursão binomial)
\end{description}


\begin{description}
	\item[Definição 2] (Função característica)
\end{description}

\begin{description}
	\item[Exemplo 8] (Subconjuntos como vectores (0,1))
\end{description}

\begin{description}
	\item[Exemplo 9] (Vectores com elementos vectores)
\end{description}


\subsection{Exercícios}

\section{Combinatória}

\input{aulas/3exercicios.tex}
\chapter{Funções}%%, funções e matrizes}

\section{Funções}

\begin{description}
	\item[Definição X] (Função)
\end{description}

\begin{description}
	\item[Exemplo X] (Funções como relações)
\end{description}

\begin{description}
	\item[Definição X] (Notção de uma linha)
\end{description}


\begin{description}
	\item[Exemplo X] (Utilizando a notação)
\end{description}

\begin{description}
	\item[Exemplo X] (Funções de contagem)
\end{description}

\begin{description}
	\item[Definição X] (Tipo de funções)
\end{description}

\begin{description}
	\item[Exemplo X] (Surjecções, injecções e bijecções como listas)
\end{description}

\begin{description}
	\item[Exemplo X] (Encriptação)
\end{description}

\begin{description}
	\item[Exemplo X] (Hashing)
\end{description}

\begin{description}
	\item[Exemplo X] (Notação de duas linhas)
\end{description}

\begin{description}
	\item[Exemplo X] (Composição de funções)
\end{description}

\begin{description}
	\item[Exemplo X] (Composição de permutações)
\end{description}

\begin{description}
	\item[Teorema X] (Permutação de cíclo)
\end{description}

\begin{description}
	\item[Exemplo X] (Utilizando a permutação de cíclo)
\end{description}

\begin{description}
	\item[Definição X] (Co-Imagem)
\end{description}

\begin{description}
	\item[Teorema X] (Estrutura de uma co-imagem)
\end{description}

\begin{description}
	\item[Exemplo X] (Partição de conjuntos)
\end{description}

\begin{description}
	\item[Exemplo X] (Contagem de funções pelo tamanho da imagem)
\end{description}

\subsection{Exercícios}

\input{aulas/4exercicios.tex}
\chapter{Grafos}
\label{cap:grafos}

Grafos são estruturas discretas que consistem em vértices, e arestas que
conectam estes vértices. Existem vários tipos de grafos, dependendo da
existência de uma direcção nas arestas, de acordo a possibilidade de várias
arestas se interligarem ao mesmo par de vértices e de acordo a existência de
\emph{loops} ou repetições.
Problemas em quase todas as disciplinas podem resolvidos utilizando modelos de
grafos. Iremos apresentar alguns exemplos para ilustrar como os grafos são
utilizados como modelos numa variedade de aéras. Por exempo, iremos mostrar como
os grafos são utilizados para representar a competição de diferentes espécias
num nicho ecológico, e como os grafos são usados para representar quem
influencia quem numa organização e etc.

Utilizando modelos de grafos, podemos determinar se é possível caminhar todas as
ruas de uma cidade sem psasar por uma rua duas vezes, e podemos determinar o
número de cores necessário para colorar as regiões de um mapa. Grafos podem ser
utilizados para determinar se um circuito pode ser implementado numa placa de
circuitos plana. Podemos distinguir entre dois compostos químicos com a mesma
fórmula molecular mas estruturas diferentes utilizando grafos. Podemos
determinar se dois computadores estão conectados por um \emph{link} de
comunicação utilizando modelos gráficos de redes. Grafos com pesos atribuidos as
suas arestas podem ser utilizados para resolver problemas como encontrar o
caminho mais curto entre duas cidades numa rede de transporte. Neste capítulo
iremos apresentar os conceitos básicos da teoria dos grafos e apresentar alguns
modelos de grafos. Para resolver uma boa parte dos problemas que podem ser
estudados utilizando grafos, iremos apresentar alguns algoritmos de grafos.
Iremos também estudar a complexidade destes algoritmos.

\section{Grafos e Modelos de Grafos}

Começamos com a definição de grafos.
\begin{defn}
\label{def51}
Um \emph{grafo} $G = (V,E)$ consiste em $V$, um conjunto não-vazio de
\emph{vértices} (ou \emph{nós}) e $E$, um conjunto de \emph{arestas}. Cada
aresta possui ou um ou mais vértices associados a esta, chamada de sua
\emph{extremidade}. Diz-se que uma aresta \emph{conecta} as suas extremidades.
\end{defn}

\begin{description}
\item[\emph{Nota}:] O conjunto de vértices $V$ de um grafo $G$ pode ser
infinito. Um grago com um conjuntos infinito de vértices ou um
número infinito de arestas é chamado de \textbf{grafo infinito}, e em comparação, um grafo com
um conjunto de finito de vértices e um conjunto finito de arestas é chamado de
\textbf{grafo finito}. Neste manual iremos considerar apenas grafos finitos.
\end{description}

Agora imagine que uma rede de computadores é formada por centros de dados e
\textit{links} de comunicação entre computadores. Podemos representar a
localização de cada centro de dados por um ponto e cada \textit{link} de
comunicação por um segmento de linha, como mostra a Figura \ref{fig51}

\begin{figure}[H]
	\centering
	\includegraphics[scale=1]{chapter/imagens/51}
	\caption{Uma Rede de Computadores.}
	\label{fig51}
\end{figure}

Esta rede de computadores pode ser modelada utilzando grafos em que os vértices
do grafo representam centros de dados e as areas representam os \textit{links}
de comunicação. No geral, visualizamos os grafos usando pontos para represtentar
os vértices e segmentos de linha, possivelmente curvos, para representar as
arestas, onde as extremidades de um segmento de linha representando uma aresta
são os pontos representando as extremidades da aresta. Quando desenhamos um
grafo, geralmente tentamos desenhar as arestas de formas a não se cruzarem. No
entanto, isto não é necessário porque qualquer representação utilizando pontos
para representar os vértices e qualquer forma de conexão entre os vértices pode
ser utilizada. De facto, existem alguns grafos que não podem ser desenhados no
plano sem que as arestas se cruzem (veja a Secção \ref{sec:107}). O ponto
principal é que a forma como desenhamos um grafo é arbritária, desde que as
conexões correctas entre os vértices estejam representadas.

Note que cada aresta do grafo representando esta rede de computadores conecta
dois vértices diferentes. Isto é, nenhuma aresta conecta um vértice a si
próprio. Além disso, duas arestas diferentes não conectam o mesmo par de
vértices. Um grafo em que cada aresta conecta dois vértices diferentes e em que
duas arestas conectam o mesmo par de vértices é chamada de \textbf{grafo
simples}. Note que num grafo simples, cada aresta está associada à um par
não-ordenado de vértices, e mais nenhuma aresta está associada a esta mesma
aresta. Consequentemente, quando existe uma aresta de um grafo simples associada
a $\{u,v\}$, também podemos dizer, sem possibilidade de confusão, que $\{u,v\}$
é uma aresta do grafo.

Uma rede de computadores pode conter múltiplas ligações entre centros de dados,
como ilustrado na Figura \ref{fig52}. Para modelar tais redes precisamos de
grafos que posuam mais de uma aresta conectando o mesmo par de vértices. Grafos
que possam ter \textbf{múltiplas arestas} a conectar os mesmos vértices são
chamados de \textbf{multigrafos}. Quando existem $m$ arestas diferentes
associadas ao mesmo par não-ordenado de vértices $\{u,v\}$, também dizemos que
$\{u,v\}$ é uma aresta de multiplicidade $m$. Isto é, podemos pensar neste
conjunto de arestas como $m$ diferentes cópias de uma aresta $\{u,v\}$.

\begin{figure}[H]
	\centering
	\includegraphics[scale=1]{chapter/imagens/52}
	\caption{Uma Rede de Computadores com Múltiplos LInks entre Centros de Dados.}
	\label{fig52}
\end{figure}

Por vezes um \textit{link} de comunicação de conecta um centro de dados consigo
próprio, possivelmente um laço de realimentação para diagnóstico. Uma rede deste
tipo é ilustrada na Figura \ref{fig53}. Para modelar esta rede precisamos
incluir arestas que conectem um vértice consigo próprio. Tais arestas são
chamadas de \textbf{laços} e as vezes podemos até ter mais de um laço no
vértice. Grafos que podem incluir laços, e possivelmente múltiplas arestas
conectando o mesmo par de vértices ou um vértice consigo próprio, são por vezes
chamados de \textbf{pseudo-grafos}.

\begin{figure}[H]
	\centering
	\includegraphics[scale=1]{chapter/imagens/53}
	\caption{Uma Rede de Computadores com Links para Diagnósticos.}
	\label{fig53}
\end{figure}


Até agora os grafos que apresentamos são \textbf{grafos não-direccionados}. As
suas arestas também são chamadas de \textbf{não direccionadas}. No entanto, para
construir um modelo de grafo, talvez achemos necessário atribuir direcções às
arestas do grafo. Por exemplo, numa rede de computadores, alguns \emph{links}
poderão operar apenas em uma direcção (tais ligaçõs são chamadas de linhas
\emph{duplex} simples). Isto pode ser o caso quando existe uma quantidade enorme
de tráfego enviada para alguns centros de dados, com pouco ou nenhum tráfego na
direcção oposta.

Para modelar tais redes de computadores utilizamos um grafo direccionado. Cada
aresta de um grafo direccionado está associada à um par ordenado. A definição de
um grafo direccionado que apresentamos aqui é mais geral do que a utilizada no
Capítulo \ref{cap:relacoes}, onde utilizamos grafos direccionados para
representar relações.

\begin{defn}
\label{def52}
Um \emph{grafo direccionado} (ou \emph{digrafo}) $(V,E)$ consiste num conjunto
não-vazio de vértices $V$ e um conjunto de \emph{arestas direccionadas} (ou
\emph{arcos}) $E$. Cada aresta direccionada está associada à um par ordenado de
vértices. A aresta direccionada associada ao par ordenado $(u,v)$ é dita que
\emph{começa} em $u$ e \emph{termina} em $v$.
\end{defn}

Quando representamos um grafo direccionado por meio de linhas, podemos utilizar
uma seta a apontar de $u$ à $v$ para indicar a direcção de uma aresta que começa
em $u$ e termina em $v$. Um grafo direccionado pode conter laços e pode conter
múltiplas arestas direccionadas que começam e terminam nos mesmos vértices. Um
grafo direccionado pode também conter arestas direccionads que conectam os
vértices $u$ e $v$ em ambas direcções; isto é, quando o dígrafo contém uma
aresta de $u$ à $v$, pode também conter uma ou mais arestas de $v$ para $u$.
Note que obtemos um grafo direccionado quando atribuímos uma direcção a cada
aresta em um grafo não-direccionado. Quando um grafo direccionado não possui
laços e não possui múltiplas arestas direccionadas, é chamado de \textbf{grafo
direccionado simples}. Como um grafo direccionado simples possui no máximo uma
aresta associada à cada par ordenado de vértices $(u,v)$, chamamos $(u,v)$ de
uma aresta se existe uma aresta associada à si no grafo.

Em algumas redes de computadores, multiplas ligações de comunicação entre dois
centros de dados podem ser representadas, como ilustrado na figura \ref{fig54}.
Grafos direccionados que possam ter \textbf{múltiplas arestas direccionadas} de
um vértice para outro vértice (possivelmente o mesmo) são usados para modelar
tais redes. Chamamos tais grafos de \textbf{multigrafos direccionados}. Quando
existm $m$ arestas direccionadas, cada associada à um par ordenado de vértices
$(u,v)$, dizemos que $(u,v)$ é uma aresta de \textbf{multiplicidade} $m$.

\begin{figure}[H]
	\centering
	\includegraphics[scale=2]{chapter/imagens/54}
	\caption{Uma Rede de Computadores Múltiplos Links de Uma Via.}
	\label{fig54}
\end{figure}

Para alguns modelos podemos necessitar um grafo onde algums arestas não são
direccionadas, enquanto que outras são direccionadas. Um grafo com ambas arestas
direccionadas e não-direccionadas é chamado de \textbf{grafo misto}. Por
exemplo, um grafo misto pode ser utilizado para modelar uma rede de computadores
que contém ligações que operam em ambas direcções e outras ligações que operam
apenas em uma direcção.

Esta terminologia para os vários tipos de grafos é sumarizada na Tabela
\ref{tab51}. Iremos algumas vezes utilizar o termo \textbf{grafo} como um termo
deral para descrever grafos com arestas direccionadas ou não direccionadas (ou
ambos), com ou sem laços e com ou sem arestas múltiplas. Em outros casos, quando
o contexto estiver claro, iremos utilizar o termo grafo para nos referirmos
apenas aos grafos não-direccionados.

\begin{table}[H]
\centering
\begin{tabular}{|l|l|l|l|}%
\toprule
\textbf{Tipo} & \textbf{Arestas} & \textbf{Múltiplas Arestas?} &
\textbf{Laços?}
\\
\midrule
Grafo simples & Não direccionada & Não & Não \\
Multigrafo & Não direccionada & Sim & Não \\
Pseudo-grafo & Nao direccionada & Sim & Sim \\
Grafo direccionado simples & Direccionada & Não & Não \\
Multigrafo direccionado & Direccionada & Sim & Sim \\
Grafo misto & Direccionada e& Sim & Sim\\
&  \qquad não direccionada & &\\
\bottomrule%
\end{tabular}%
\caption{Terminologia dos Grafos.}
\label{tab51}
\end{table}

Por causa do recente interesse na teoria dos gradfos, e por causa da sua
aplicação à uma variedade de disciplinas, muitas terminologias da teoria dos
grados foram introduzidas. O estudante deverá determinar como tais termos estão
a ser utilizados quando os encontrar na literatura. A terminologia utilizada por
matemáticas para descrever grafos tem sido padronizada incrementalmente, mas a
terminologia usada em outras disciplinas ainda é muito variada. Embora a
terminologia usada para descrever grafos pode variar, três questões nos ajudam a
entender a estrutura de um grafo:
\begin{itemize}
  \item As arestas do grafo são não-direccionadas ou direccionadas (ou ambas)?
  \item Se o grafo é não-direccionado, existem múltiplas arestas que conectam o
  mesmo par de vértices? Se o grafo é direccionado, existem múltiplas arestas
  direccionads?
  \item Existem laços?
\end{itemize}

Responder a estas questões ajuda-nos a entender grafos independemente da
terminologia particular utilizada.


\subsection*{\underline{Modelos de Grafos}}

Grafos são utilizados numa enorme variedade de modelos. Iniciamos esta secção
por descrever como construir modelos de redes de comunicação que ligam centros
de dados. Iremos completar a secção por descrever alguns modelos diversos de
grados para algumas aplicações interessantes. Iremos retornar à algumas dessas
aplicações mais no final do capítulo. Iremos introduzir modelos adicionais de
grafos em secções subsequentes.

\begin{description}
\item[REDES SOCIAIS] Grafos são extensivamente utilizados para modelar
estruturas sociais baseadas nos diferentes tipos de relacionamentos entre
pessoas ou grupos de pessoas. Estas estruturas sociais, e os grafos que as
representam, são chamadas de \textbf{redes sociais}. Nestes modelos de grafos,
indivíduos ou organizaçãoes são representados por vértices; relacionamentos
entre indivíduos ou organizaçãoes são representadas por arestas. O estudo de
redes sociais é uma área multidisciplinar extremamente activa, e muitos
diferentes tipos de relacionamento entre pessoas foram estudados utilizando as
mesmas. Iremos apresentar algumas das mais geralmente estudadas redes sociais.

\begin{exmp}
\label{exem51}
\textbf{Grafos de Relação Pessoal e de Amizade} Podemos utilizar um grafo
simples para representar o facto de duas pessoas se conhecerem ou não, isto é, se
têm uma relação pessoal ou se são amigos (no mundo real ou no mundo
virtual através de uma rede social como o Facebook). Cada pessoa num grupo
particular de pessoas é representada por um vértice. Uma aresta
não-direccionada é usada para conectar duas pessoas quando estas pessoas
conhecem-se, ou seja têm uma relação pessoal ou se são amigos.
Não são utilizadas múltiplas arestas nem laços (se quisermos introduzir o
conceito de auto-conhecimento, acrescentaríamos laços). Um exemplo de grafo de
relação pessoal é apresentado na Figura \ref{fig55}. O grafo de relação pessoal
de todas as pessoas no mundo possui mais de seis biliões de vértices e
provavelmente mais de um trilião de arestas!
\end{exmp}

\begin{figure}[H]
	\centering
	\includegraphics[scale=1.5]{chapter/imagens/55}
	\caption{Um Grafo de Relação Pessoal.}
	\label{fig55}
\end{figure}

\tikz \graph [layout=layered, radius=1cm] {
k -- j -- p -- e;
j -- l -- jl -- kr;
p -- t -- k -- c;
p -- k -- a -- s -- kk;
t -- s -- sq;
p -- lz;
a -- lz;
};

\item[REDES DE COMUNICAÇÃO]
\item[REDES DE INFORMAÇÃO]
\item[APLICAÇÕES PARA O DESENHO DE SOFTWARE]
\item[REDES DE TRANSPORTE]
\item[REDES BIOLÓGICAS]
\item[TORNEIOS]
\end{description}

\section{Terminologia dos Grafos e Tipos de Grafos Especiais}
\subsection*{\underline{Introdução}}
\subsection*{\underline{Terminologia Básica}}
\subsection*{\underline{Alguns Grafos Simples Especiais}}
\subsection*{\underline{Grafos Bipartidos}}
\subsection*{\underline{Grafos Bipartidos e Combinações}}
\subsection*{\underline{Algumas Aplicações dos Tipos de Grafos Especiais}}
\subsection*{\underline{Novos Grafos à Partir de Grados Antigos}}

\section{Representação de Grafos e Isomorfismo de Grafos}
\subsection*{\underline{Introdução}}
\subsection*{\underline{Representação de Grafos}}
\subsection*{\underline{Matrizes de Adjacência}}
\subsection*{\underline{Matrizes de Incidência}}
\subsection*{\underline{Isomorfismo de Grafos}}
\subsection*{\underline{Determinando se Dois Grafos Simples são Isomórficos}}

\section{Conectividade}
\subsection*{\underline{Introdução}}
\subsection*{\underline{Trajectória}}
\subsection*{\underline{Conectividade de Grafos Não-Direccionados}}
\subsection*{\underline{O Quão Conectado é Um Grafo}}
\subsection*{\underline{Conectividade de Grafos Direccionados}}
\subsection*{\underline{Trajectória e Isomorfismo}}
\subsection*{\underline{Contando a Trajectória entre Vértices}}

\section{Trajectória de Euler e de Hamilton}

\section{Problemas do Caminho-Mais-Curto}

\section{Grafos Planares}

\section{Coloração de Grafos}










\input{aulas/5exercicios.tex}
\chapter{Relações}

Relações entre elementos de conjuntos ocorrem em muitos contextos. Todos dias lidamos com relações como por 
exemplo: uma pessoa e o seu número de telemóvel, um empregado(a) e o seu salário, etc. Em matemática estudamos
relações como as que existem entre um número inteiro positivo e um seu divisor, um inteiro e o seu quadrado, um 
valor real $x$ e o valor $f(x)$ onde $f$ é uma função, etc.

As relações são representadas utilizando uma estrutura chamada de \emph{relação}, que é simplesmente um subconjunto
do produto cartesiano de conjuntos. As relações podem ser utilizadas para resolver problemas tais como: determinar quais
pares de cidade estão ligadas pela mesma companhia área numa rede, armazenamento de informações em bases de dados, etc.

\section{Relações e suas propriedades}


A forma mais directa de expressar uma relação entre elemento de dois conjuntos é por utilizar pares ordenados (dois elementos).
Por esta razão, os conjuntos de pares ordenados são chamados de \emph{relações binárias}. Nesta secção apresentamos
a terminologia básica utilizada para descrever as relações binárias.

\begin{description}
	\item[Definição 1] Sejam $A$ e $B$ conjuntos, uma relação binária de $A$ para $B$ é um subconjunto de $A \times B$
\end{description}

Por outras palavras uma relação binária de $A$ para $B$ é um conjunto $R$ de pares ordenados onde o primeiro elemento
de cada par ordenado provem de $A$ e o segundo elemento provem de $B$. Utilizamos a notação $a R b$ para denotar 
que $(a,b) \in R$ e $a \not{R} b$ para denotar que $(a,b) \notin R$. Além disso, quando $(a,b)$ pertecem a $R$, dizemos
que $a$ \emph{está relacionado} a $b$ por intermédio de $R$.

As relações binárias representam relacionamentos entre elementos de dois conjuntos. Apresentaremos mais adiante 
as relações n-árias que expressam relacionamentos entre elementos de mais de dois conjuntos. Iremos omitir a palavra
\emph{binária} sempre que não houver perigo de má interpretação. Os exemplos a seguir ilustram o conceito de \emph{relação}.

\begin{description}
	\item[Exemplo 1] {Sejam $A$ o conjunto dos estudantes da tua escola, e $B$ o conjunto das disciplinas. Seja $R$ a 
	a relação que consite nos pares $(a,b)$, onde $a$ é um estudante inscrito na disciplina $b$. Por exemplo, se João e David
	estão inscritos na disciplina de Estruturas Discretas (ED), os pares (João, ED) e (David, ED) pertecem a $R$. Note que
	se David não está inscrito na disciplina de ED, então o par (David, ED) não pertence a $R$. Se um estudante
	não está inscrito em nenhuma disciplina, não existirá nenhum par em $R$ com este estudante como primeiro elemento. Da
	mesma forma se uma disciplina não existe não existirá nenhum par em $R$ com esta discplina como segundo elemento.}
\end{description}

\begin{description}
	\item[Exemplo 2] {Sejam  $A = \{0,1,2\}$ e $B = \{a,b\},$ então $\{(0,a), (0,b), (1,a), (2,b)\}$ é uma relação
	de $A$ para $B$. Isto significa que, por exemplo, $0 R a$, mas $1 \not{R} b$.}. Continua
\end{description}

\section{Relações n-árias}

%Aula de quarta feira
\section{Representação de relações}

\subsection{Introdução}

\textbf{Nota}: Nesta secção, utilizaremos sómente as relações binárias. Por esta razão a palavra relação irá apenas referir-se a
relações bináris. 

Existem muitas formas de representar uma relação entre conjuntos finitos. Uma forma é por listar os pares ordenados.
Outra forma de representar uma relação é por meio de tabelas como vimos na secção anterior. Nesta secção vamos apresentar
dois métodos de representação alternativos: matrizes zero-um e gráfos direccionados. No geral, as matrizes são apropriadas
para a representação de relações em programas de computador. Por outro lado, algumas pessoas acham a representação
de relações utilizando grafos direcconados mais útil ao entendimento das propriedades dessas relações.

\subsection{Representação de relações por meio de matrizes}

A relação entre conjuntos finitos pode ser representada utilizando matrizes zero-um. Suponha que $R$ é uma relação
de $A = \{a_1, a_2, \ldots, a_m\}$ para $B = \{b_1, b_2, \ldots, b_n\}$. (Aqui os elementos dos conjuntos A e B são listados
duma forma particular, embora arbitrária. Além dos maais, quando $A = B$ utilizamos a mesma ordenação para $A$ e $B$.)
A relação $R$ pode ser representada pela matriz $M_R = [m_{ij}]$,\\


$m_{ij} = \begin{cases}
	1$ se $(a_i, b_j) & \in R,\\
	0$ se $(a_i, b_j) & \notin R.
\end{cases}$


Por outras palavras, a matriz zero-um que representa $R$ tem o valor 1 em $(i,j)$ quando $a_i$ está relacionado
a $b_j$, e o valor 0 nesta posição se $a_i$ não está relacionado a $b_j$. Esta representação depende da ordem utilizada
para $A$ e $B$. A utilização de matrizes para representar relações é ilustrados no exemplos a seguir.

\begin{description}
	\item[Exemplo 1] {Suponha que $A = \{1,2,3\}$ e $B = \{1,2\}$. Seja R a relação de $A$ para $B$ contendo os pares
	$(a,b)$ se $a \in A$, $b \in B$ e $a > b$. Qual é matriz que representa $R$ se $a_1 = 1$, $a_2 = 2$, $a_3 = 3$ e 
	$b_1 = 1$ e $b_2 = 2?$}
\end{description}

\emph{Solução:} Como $R = \{(2, 1), (3, 1), (3, 2)\}$, a matriz para $R$ é:

\[
M_R = \begin{bmatrix}
	0 & 0\\
	1 & 0\\
	1 & 1
\end{bmatrix}
\]

Os 1s em $M_R$ mostram que os pares $(2,1), (3,1)$ e $(3,2)$ pertencem a $R$. Os 0s mostram que os outros pares não
pertencem a $R$.

\begin{description}
	\item[Exemplo 2]{Sejam $A = \{a_1, a_2, a_3\}$ e $B = \{b_1, b_2, b_3, b_4, b_5\}$, quais pares ordenados estão na
	relação $R$ representada pela matriz}
\end{description}

\[
	M_R = \begin{bmatrix}
	0 & 1 & 0 & 0 & 0\\
	1 & 0 & 1 & 1 & 0\\
	1 & 0 & 1 & 0 & 1
	\end{bmatrix}?
\]
	
\emph{Solução} Como $R$ consiste nos pares ordenados $(a_i, b_j)$ com $m_{ij} = 1$ daí resulta que
$R = \{(a_1,b_2), (a_2,b_1), (a2, b3), (a2, b4), (a3, b1), (a3, b3), (a3, b5)\}$

A matriz de uma relação em um conjunto, que é uma matriz quadrada, pode ser utilizada para determinar se a relação
possui certas propriedades. Sabemos que uma relação $R$ num conjunto $A$ é reflexiva se $(a,a) \in R$ sempre que
$a \in A$, então, $R$ é reflexiva se e somente se $(a_i, a_i) \in R$ para $i = 1,2,\ldots,n$. Assim, $R$ é reflexiva
se e somente se $m_{ii} = 1$, para $i = 1,2,\ldots,n$. Por outras palavras, $R$ é reflexiva se todos os elementos
da diagonal principal de $M_R$ são iguais a $1$, como ilustrado na Figura \ref{Figura61}. Note que os elementos fora da diagonal
podem ser $0s$ ou $1s$.

\begin{figure}[H]
	\centering
	\[
	\begin{bmatrix}
	 1	& 	& 	&	&	&	&	&\\
	 	& 1 &	&	&	&	&	&\\
		&	& 1 &	&	&	&	&\\
		&	&  	& .	&	&	&	&\\
		&	&	& 	& . &	&	&\\
		&	&	&	&	& . &	&\\
		&	&	& 	&	&	& 1	&\\
		&	&	&	&	&	&	& 1 
	\end{bmatrix}
	\]
	\caption{A matriz zero-um para uma relação reflexiva. (Os elementos fora da diagonal podem ser 0 ou 1.)}
	\label{Figura61}
\end{figure}

A relação $R$ é simétrica se $(a,b) \in R$ implica que $(b, a) \in R$. Consequentemente, a relação $R$ no conjunto
$A = \{a_1, a_2, \ldots, a_n\}$ é simétrica se e somente se $(a_j, a_i) \in R$ sempre que $(a_i, a_j) \in R$. Em termos
dos valores de $M_R$, $R$ é simétrica se e somente se $m_{ji} = 1$ sempre que $m_{ij} = 1$. Isto também significa que 
$m_{ji} = 0$ sempre que $m_{ij} = 0$. Consequentemente, $R$ é simétrica se e somente se $m_{ij} = m_{ji}$, para todos
os pares de inteiros $i$ e $j$ com $i = 1,2,\ldots,n$ e $j=1,2,\ldots,n$. $R$ é simétrica se e somente se $M_R = (M_R)^t$
onde $(M_R)^t$ é matriz transposta de $M_R$.


A relação $R$ é antissimétrica se e somente se $(a, b) \in R$ e $(b, a) \in R$ implica que $a = b$. Consequentemente,
a matrix de uma relação antissimétrica tem a propriedade de que se $m_{ij} = 1$ com $i \ne j$, então $m_{ji} = 0$.
Ou, em outras palavras, $m_{ij} =0$ ou $m_{ji} =0$ quando $i \ne j$. A forma da matriz para uma relação antissimétrica
é ilustrada na Figura \ref{Figura62}.

\begin{figure}[H]
	\centering
	\includegraphics[scale=0.75]{aulas/imagens/62}
	\caption{A matriz zero-um para uma relação simétrica e antissimétrica.}
	\label{Figura62}
\end{figure}

Suppose that the relation R on a set is represented by the matrix

\begin{description}
	\item[Exemplo 3]{Suponha que a relação $R$ é representada pela matriz}
\end{description}
\[
	M_R = \begin{bmatrix}
	1 & 1 & 0\\
	1 & 1 & 1\\
	0 & 1 & 1
	\end{bmatrix}?
\]

$R$ é reflexiva, simétrica e/ou antisimétrica?

\emph{Solução:} Como todas os elementos das diagonais nesta matriz são iguais a $1$, $R$ é reflexiva.
Além do mais, como $M_R$ é simétrica, então $R$ é simétrica. É também fácil notar que $R$ não é antissimétrica.


As operações booleanas estudados anteriormente também podem ser utilizados para encontrar as matrizes que 
representam a união e a intersecção de duas relações. Suponha que $R_1$ e $R_2$ são relações num conjunto $A$
representada pelas matrizes $M_{R_1}$ e $M_{R_2}$, respectivamente. A matriz que representa a união destas duas
relações possui o valor $1$ nas posições em que  $M_{R_1}$ ou $M_{R_2}$ possuem o valor $1$. A matriz que representa a
intersecção destas duas relações possui o valor $1$ nas posições em que  $M_{R_1}$ e $M_{R_2}$ possuem o valor $1$.
Sendo assim, as matrizes que representam a união e a intersecção destas duas relações são

$M_{R_1 \cup R_2} = M_{R_1} \lor M_{R_2}$ e $M_{R_1 \cap R_2} = M_{R_1} \land M_{R_2}$

\subsection{Representação de relações por meio de grafos direccionados}

Vimos anteriormente que uma relação pode ser representada por uma listagem de todos os seus pares ordenados ou por meio
de uma matriz zero-um. Existe outra forma importante de representar uma relação utilizando uma representação 
pictural. Cada elemente do conjunto é representado por um ponto, e cada par ordenado é representado utilizando um arco
cuja direcção indicada por uma seta. Utilizamos essa representação sempre que utilizamos uma relação num conjunto finito como
um grafo direccionado ou dígrafos.


\begin{description}
	\item[Definição 1]{Um \emph{grafo direccionado}, ou \emph{dígrafo}, consiste num conjunto $V$ de \emph{vértices} 
	(ou \emph{nós}) e um conjunto $E$ pares ordenados dos elementos de $V$ chamados de \emph{arestas} (ou \emph{arcos}). 
	O vértice $a$ é chamado de \emph{vértice inicial} da aresta $(a,b)$, e o vértice $b$ é chamado de \emph{vértice terminal}
	desta aresta.}
\end{description}

Uma aresta da forma $(a, a)$ é representada utilizando um arco do vértice $a$ de volta à sí mesmo. Tal aresta é chamada de 
\textbf{laço} ou \textbf{loop}.


\begin{description}
	\item[Exemplo 4]{O grafo direccionado com os vértices $a, b, c$ e $d$ e as arestas $(a,b), (a,d), (b,b), (b,d), (c,a)
	(c,b)$ e $(d,b) é apresentado na Figura \ref{Figura63}$
	}
\end{description}

\begin{figure}[H]
	\centering
	\includegraphics[scale=0.75]{aulas/imagens/63}
	\caption{Um grafo direccionado.}
	\label{Figura63}
\end{figure}


A relação $R$ no conjunto $A$ é representada pelo grafo ordenado que possui elementos de $A$ como seus vértices e os pares 
ordenados $(a,b)$, onde $(a,b) \in R$, como arestas. Esta atribuição configura uma correspondência um-para-um entre as relações
no conjunto $A$ e os grafos direcionados que possuem $A$ como o seu conjunto de vértices. Assim, cada afirmação sobre relações
corresponde a uma afirmação sobre grafos direccionados, e vice-versa. Grafos direccioandos fornecem uma exibição visual das
relações e por isso são utilizados no estudo das relações e de suas propriedades. A utilização de grafos direccionados 
na represetntação de relações num conjunto é ilustrada nos seguintes exemplos.

The directed graph of the relation

on the set {1, 2, 3, 4} is shown in Figure 4.


\begin{description}
	\item[Exemplo 5]{
	O grafo direccionado da relação\\
	$R = {(1, 1), (1, 3), (2, 1), (2, 3), (2, 4), (3, 1), (3, 2), (4, 1)}$\\
	no conjunto ${1, 2, 3, 4}$ é ilustrado na Figura \ref{Figura64}
	}
\end{description}

\begin{figure}[H]
	\centering
	\includegraphics[scale=0.75]{aulas/imagens/64}
	\caption{Um grafo direccionado.}
	\label{Figura64}
\end{figure}

\begin{description}
	\item[Exemplo 6]{
	Quais são os pares ordenados na relação $R$ que representada pelo grafo direccionado da Figura \ref{Figura65}}
\end{description}

\begin{figure}[H]
	\centering
	\includegraphics[scale=0.75]{aulas/imagens/65}
	\caption{Um grafo direccionado da relação R.}
	\label{Figura65}
\end{figure}

\emph{Solução:} Os pares ordenados $(x,y)$ na relação na relação são\\
$R = {(1, 3), (1, 4), (2, 1), (2, 2), (2, 3), (3, 1), (3, 3), (4, 1), (4, 3)}$

Cada um destes pares corresponde à uma aresta do grafo direccionado sendo $(2,2)$ e $(3,3)$ dois laços.

O grafo direccionado que representa uma relação pode ser utilizado para determinar se a relação possui certas propriedades.
Por exemplo, a relação é reflexiva se e somente se existe um laço em cada vértice do grafo direccionado, de tal forma que
todos os pares ordenados da forma $(x,x)$ ocorrem na relação. A relação é simétrica se e somente se para cada aresta entre vértices
distintos no digrafo existe uma aresta na direcção oposta, tal que $(y,x)$ existe na relação sempre que $(x,y)$ existe na
relação. Da mesma forma, uma relação é antissimétrica se e somente se não existem duas arestas
em direcções opostas entre vértices distintos. Finalmente, uma relação é transitiva se e somente se sempreque
que existe uma aresta de um vértice $x$ para um vértice $y$ e uma aresta de um vértice $y$ para um vértice $z$, existe uma
aresta de $x$ para $z$ (completando um triângulo onde cada lado é aresta direccionada correctamente).


\begin{description}
	\item[Exemplo 7]{Determine se os grafos direccionados apresentados nas Figuras \ref{Figura66} e \ref{Figura67} são
	reflexivos, simétricos, antissimétricos e/ou transitivos}.
\end{description}

\begin{figure}[H]
	\centering
	\includegraphics[scale=0.75]{aulas/imagens/66}
	\caption{Um grafo direccionado da relação R.}
	\label{Figura66}
\end{figure}

\begin{figure}[H]
	\centering
	\includegraphics[scale=0.75]{aulas/imagens/67}
	\caption{Um grafo direccionado da relação S.}
	\label{Figura67}
\end{figure}

\emph{Solução:} Como existem laços em todos os vértices do grafo direccionado de $R$, ele é reflexivo. $R$ não é simétrica
nem anti-simétrica porque existe uma aresta de $a$ para $b$ mas não de $b$ para $a$, mas existem arestas em ambas direcções
que conectam $b$ e $c$. Finalmente, $R$ não é transitiva porque existe uma aresta de $a$ para $b$ e uma aresta de $b$ para $c$,
mas não existe uma aresta de $a$ para $c$. Como não existem laços em todo os vértices do grafo direccionado de S, esta relação não é reflexiva. É simétrica
mas não antissimétrica, porque cada aresta entre vértices distintos é acompanhada por uma aresta na direcção oposta. Não é
díficil notar também que o grafo direccionado de $S$ não é transitivo, porque $(c,a)$ e $(a,b)$ pertencem a $S$, mas
$(c,b)$ não pertence a $S$.

Estudaremos os grafos em mais detalhnes no próximo capítulo.


\section{Fechamento de relações}

\subsection{Introdução}

Uma rede de computadores de uma empresa possui centros de dados em Benguela, Bengo, Cabinda, Cunene, Huíla e Luanda.
Existem ligações directas de Benguela para o Bengo, de Benguela para o Bengo, de Benguela para Cunene, do Bengo para o
Cunene, de Cunene para Cabinda e da Huíla para Luanda. Seja $R$ a relação que contém $(a,b)$ se existe uma ligação do 
centro de dados em $a$ com o centro de dados em $b$, como podemos determinar se existe uma ligação (possívelmente indirecta)
composta de uma ou mais linhas de um centro para outro? Como nem todas as ligações são directas, tal como a ligação de
Benguela para Cabinda que passa por Cunene, $R$ não pode ser utilizada directamente para responder esta questão.
Na linguagem das relações, $R$ não é transitiva, portanto não contém todos os pares que podem ser ligado. Tal como iremos
mostrar nesta secção, é possível achar todos os pares de centros de dados que possuem um link por construír
uma relação transitiva $S$ que contém $R$ tal que $S$ é um subconjunto de todas as relações transitivas que contêm $R$.


\section{Relações de equivalência}

\section{Ordens parciais}

\subsection{Exercícios}
\chapter*{Relações}
%%1

\section*{Relações e suas propriedades}

For each of these relations on the set {1, 2, 3, 4}, decide
whether it is reflexive, whether it is symmetric, whether
it is antisymmetric, and whether it is transitive.

\begin{enumerate}
  	\item {Liste os pares ordendados na relação $R$ de $A = \{0,1,2,3,4\}$ para $B = \{0,1,2,3\}$ onde $(a,b) \in R$ se
  	e somente se}
  	\begin{enumerate}
  	  	\item $a = b$ \item $a + b = 4$ \item $a > b$ \item $a | b$ \item $mdc(a,b)=1$ \item $mmc(a,b)=2$
  	\end{enumerate}
  	
  	\item{Para cada uma das relações no conjunto $\{1,2,3,4\}$ indiqye se são: reflexivas, simétricas, antissimétricas
  	e transitivas}
  	\begin{enumerate}
  	  	\item $\{(2, 2), (2, 3), (2, 4), (3, 2), (3, 3), (3, 4)\}$
  	  	\item $\{(1, 1), (1, 2), (2, 1), (2, 2), (3, 3), (4, 4)\}$
  	  	\item $\{(2, 4), (4, 2)\}$
  	  	\item $\{(1, 2), (2, 3), (3, 4)\}$
  	  	\item $\{(1, 1), (2, 2), (3, 3), (4, 4)\}$
  	  	\item $\{(1, 3), (1, 4), (2, 3), (2, 4), (3, 1), (3, 4)\}$
  	\end{enumerate}
 a is taller than b.
b) a and b were born on the same day.
c) a has the same first name as b.
d) a and b have a common grandparent.

  	\item {Determine se a relação $R$ conjunto de todas as pessoas é reflexiva, simétrica, antissimétrica, e/ou
  	transitiva, onde $(a,b) \in R$ se e somente se}
  	\begin{enumerate}
  		\item $a$ é mais alto que $b$ \item $a$ e $b$ onde 
  	\end{enumerate}	
\end{enumerate}

\section*{Representação de relações}

\begin{enumerate}
  	
\end{enumerate}

\vspace*{2em}
\begin{center}\textbf{Continua!!! (Mais exercícios)}\end{center}
\input{aulas/7.tex}
\input{aulas/7exercicios.tex}
\input{aulas/8.tex}
\input{aulas/8exercicios.tex}
\input{aulas/cf.tex}

\end{document}