\chapter{Conjuntos e combinatória}

\section{Conjuntos}

\begin{description}
	\item[Definição 1] (Notação de conjuntos)
\end{description}

\subsection{Propriedades dos conjuntos}

\begin{description}
	\item[Teorema 1] (Regras algebraicas dos conjuntos)
\end{description}

\begin{description}
	\item[Exemplo 1] (Diagramas de Venn e demostrações de equações de conjuntos)
\end{description}

\begin{description}
	\item[Exemplo 2] (Métodos elementares de demonstração de equações de conjuntos)
\end{description}

\begin{description}
	\item[Exemplo 3] (O método tabular para demostrações de equações de conjuntos)
\end{description}

\begin{description}
	\item[Exemplo 4] (Demonstração algebraica)
\end{description}

\subsection{Ordenação de conjuntos}

\begin{description}
	\item[Exemplo 5] (Ordem lexicográfica)
\end{description}

\begin{description}
	\item[Exemplo 6] (Ordenação por dicionário em palavras ou cadeia de caracteres)
\end{description}

\subsection{Subconjuntos}

\begin{description}
	\item[Teorema 2] (Fórmula do binómio coeficiente)
\end{description}

\begin{description}
	\item[Teorema 3] (Número de listas ordenadas)
\end{description}

\begin{description}
	\item[Exemplo 7] (Recursão binomial)
\end{description}


\begin{description}
	\item[Definição 2] (Função característica)
\end{description}

\begin{description}
	\item[Exemplo 8] (Subconjuntos como vectores (0,1))
\end{description}

\begin{description}
	\item[Exemplo 9] (Vectores com elementos vectores)
\end{description}


\subsection{Exercícios}

\section{Combinatória}
