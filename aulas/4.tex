\chapter{Funções}%%, funções e matrizes}

\section{Funções}

\begin{description}
	\item[Definição X] (Função)
\end{description}

\begin{description}
	\item[Exemplo X] (Funções como relações)
\end{description}

\begin{description}
	\item[Definição X] (Notção de uma linha)
\end{description}


\begin{description}
	\item[Exemplo X] (Utilizando a notação)
\end{description}

\begin{description}
	\item[Exemplo X] (Funções de contagem)
\end{description}

\begin{description}
	\item[Definição X] (Tipo de funções)
\end{description}

\begin{description}
	\item[Exemplo X] (Surjecções, injecções e bijecções como listas)
\end{description}

\begin{description}
	\item[Exemplo X] (Encriptação)
\end{description}

\begin{description}
	\item[Exemplo X] (Hashing)
\end{description}

\begin{description}
	\item[Exemplo X] (Notação de duas linhas)
\end{description}

\begin{description}
	\item[Exemplo X] (Composição de funções)
\end{description}

\begin{description}
	\item[Exemplo X] (Composição de permutações)
\end{description}

\begin{description}
	\item[Teorema X] (Permutação de cíclo)
\end{description}

\begin{description}
	\item[Exemplo X] (Utilizando a permutação de cíclo)
\end{description}

\begin{description}
	\item[Definição X] (Co-Imagem)
\end{description}

\begin{description}
	\item[Teorema X] (Estrutura de uma co-imagem)
\end{description}

\begin{description}
	\item[Exemplo X] (Partição de conjuntos)
\end{description}

\begin{description}
	\item[Exemplo X] (Contagem de funções pelo tamanho da imagem)
\end{description}

\subsection{Exercícios}
